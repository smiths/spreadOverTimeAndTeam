\documentclass[10pt, conference]{IEEEtran}

\usepackage[pdftex]{graphicx}
\usepackage[cmex10]{amsmath}
\usepackage{amsfonts}

\usepackage{color}
\usepackage{hyperref}
\hypersetup{colorlinks=true,
    linkcolor=blue,
    citecolor=blue,
    filecolor=blue,
    urlcolor=blue,
    unicode=false}
\urlstyle{same}

\usepackage{tabularx}
\usepackage{booktabs}
\usepackage{siunitx}
\usepackage{subfig}
\usepackage{paralist}
\usepackage{colortbl}
\usepackage{listings}

\usepackage{enumitem}
\usepackage{pgfplots}
\usepackage{pgfplotstable} % For importing data from a csv

% Allows us to exclude some bars from certain plots, to allow multicoloured plots
\pgfplotsset{
    discard if/.style 2 args={
        x filter/.code={
            \edef\tempa{\thisrow{#1}}
            \edef\tempb{#2}
            \ifx\tempa\tempb
                \def\pgfmathresult{inf}
            \fi
        }
    },
    discard if not/.style 2 args={
        x filter/.code={
            \edef\tempa{\thisrow{#1}}
            \edef\tempb{#2}
            \ifx\tempa\tempb
            \else
                \def\pgfmathresult{inf}
            \fi
        }
    }
}

\newif\ifcomments\commentstrue

\ifcomments
\newcommand{\authornotation}[3]{\textcolor{#1}{[#3 ---#2]}}
\newcommand{\todo}[1]{\textcolor{red}{[TODO: #1]}}
\else
\newcommand{\authornotation}[3]{}
\newcommand{\todo}[1]{}
\fi

\newcommand{\wss}[1]{\authornotation{blue}{SS}{#1}}
\newcommand{\ms}[1]{\authornote{cyan}{MS}{#1}}

\newcommand{\progname}{SFS}
\newcommand{\colAwidth}{0.13\textwidth}
\newcommand{\colBwidth}{0.84\textwidth}

\begin{document}

\title{A Software Engineering Capstone Infrastructure that Encourages Spreading
Work Over Time and Team}

\author{\IEEEauthorblockN{}
\IEEEauthorblockA{}

% \author{\IEEEauthorblockN{Spencer Smith, Christopher Schankula, Lucas Dutton and Christopher Anand}
% \IEEEauthorblockA{Computing and Software Department\\
% McMaster University, Canada\\
% Email: smiths@mcmaster.ca, schankuc@mcmaster.ca, duttonl@mcmaster.ca, anandc@mcmaster.ca}

% \and
% \IEEEauthorblockN{Sumanth Shankar}
% \IEEEauthorblockA{Mechanical Engineering Department\\
% McMaster University, Canada\\
% Email: shankar@mcmaster.ca }
}

\maketitle
  
\begin{abstract}

How can instructors facilitate spreading out the work in a software engineering
or computer science capstone course across time and among team members?
Currently teams often compromise the quality of their learning experience by
frantically working before each deliverable.  Some team members further
compromise their own learning, and that of their colleagues, by not contributing
their fair share to the team effort. To mitigate these problems, we propose
using a GitHub template that contains all the initial infrastructure a team
needs, including the folder structure, text-based template documents and
template issues. In addition, we propose each team begins the year by
identifying specific quantifiable individual productivity metrics for
monitoring, such as the count of meetings attended, issues closed and number of
commits.  Initial data suggests that these steps may have an impact.  In 2022/23
we observed 24\% of commits happening on the due dates.  After partially
introducing the above ideas in 2023/24, this number improved to 18\%.  To
measure the fairness we introduce a fairness measure based on the disparity
between number of commits between all pairs of teammates.  Going forward we
propose an experiment where commit data and interview data is compared between
teams that use the proposed interventions and those that do not.

\end{abstract}

\begin{IEEEkeywords}
software engineering; capstone; template repository; productivity measures;
fairness metric
\end{IEEEkeywords}

\section{Introduction} \label{SecIntro}

The workload for a software engineering or computer science capstone team
project is often unevenly distributed over time and among team members.  Teams
typically work in frantic bursts of activity right before a deadline and then
cease almost all activity until their next deadline.  These work habits
compromise the learning objectives of the course because the students do not
have time to properly plan their activities or reflect on their work.  The
uneven distribution of effort among team mates is also problematic.  Some
students take on an unfair share of the work, causing them stress and possibly
hurting their experience in other courses, while those investing less effort
(so-called free riders~\cite{tushevUsingGitHubLarge2020}) miss important
learning opportunities.  How can instructors mitigate these problems?

To address the uneven distribution of work, we need to first think about why the
problems exist.  A student project is not the same environment as the workplace.
Students are often learning the content just before applying their knowledge.
Since they are doing a capstone project for the first time, students might
struggle with determining the expectations; they might not know where to start.
In industry, team members usually dedicate most of their time to a project, unlike
an academic environment where students are juggling many
courses~\cite{connReusableAcademicstrengthMetricsbased2004}. Moreover, team
members cannot be fired or moved to another project for poor performance.  Peer
pressure and the prospect of uncomfortable social interactions can make it
challenging for someone to take charge of their group, or to criticize other
group members. %any citations on these challenges?

% What can instructors do to help the students? How can they encourage the
% students to focus their efforts?  Instructors control the organization,
% structure and content of their course, and they control the rules for awarding
% grades to students. An uneven workload distribution over time and among
% team-members can potentially be improved by the instructor providing as much of
% the mundane project infrastructure as possible to save student time and to
% clearly show expectations. Furthermore, instructors can use grades to encourage
% participation by all team members.

To save student effort and to make expectations clear, we suggest
requiring all capstone teams to start from the same GitHub template repository.
The template repository is populated with folders, text-based template documents
and template issues.  [ADD Citations on this topic]. Too much freedom in decision
making can be paralyzing for a new team.  They don't know where to begin.  By
making the infrastructure decisions for them, they can focus on their project.

An uneven workload distribution among teammates can be improved by early
awareness of potential problems. Ideally, the teams should plan how to deal with
problems, before the problems occur.  We propose doing this by identifying
quantifiable productivity metrics for each team member (like counting meetings
attended, issues closed, and commits) and having the team write a team charter
at the beginning of the term that lays out their unambiguous expectations.  The
idea of a team charter is not new \cite{mathieuLayingFoundationSuccessful2009,
johnsonTeamChartersHelp2022, hughstonEmpiricalStudyTeam2013}, but as far as we
are aware, we are the first to suggest incorporating specific quantifiable
GitHub-derived metrics and consequences. Other studies have used commits to
understand/explain team behaviour a
posteriori~\cite{gitinabardStudentTeamworkProgramming2020,
tushevUsingGitHubLarge2020}, but having the students actively collect and use
this data during the course appears to be a new idea.

In Section~\ref{SecInfrastruct} we describe the baseline structure of the
capstone course.  We propose two interventions to the baseline: 1) using a
template repository; and, 2) explicit quantifiable team contribution
measurement. We follow this with some encouraging preliminary data from when we
partially introduced the two interventions into the course
(Section~\ref{SecPrelimData}).  We then describe our proposed approach for
collecting more detailed data to judge the effectiveness of the proposed
interventions (Section~\ref{SecProposedExperiment}). The presentation of the
proposed experiment includes discussion of threats to validity.

\section{Baseline and Proposed Infrastructure} \label{SecInfrastruct}

The infrastructure described here matches the final year SE capstone course at
[Redacted]. %McMaster University The course follows the ACM guidelines of
spanning a full year, being a group project, having an implementation as its end
deliverable, having a customer for each project, and including student
reflection~\cite{ACM2015}.  This course is currently delivered to 150 students
divided into 29 groups of 4--5 members (the typical size for capstone
courses~\cite{tenhunenSystematicLiteratureReview2023}).  Teams are provided with
a list of curated software development projects from academia and industry.
Teams can also propose their own projects.  Most projects have a
supervisor/client that the team can meet with to discuss their project.  In
cases where there is no supervisor, explicitly identifies the relevant
stakeholders/users for their project.

\subsection{Structure and Timeline} \label{Sec_Structure}

Figure~\ref{Fig_VModel} show the V-model~\cite{ForsbergAndMooz1991} structure of
the capstone course. The documents created include a Software Requirements
Specification (SRS) and Verification and Validation (VnV) plans and reports. Due
to time constraints, not all artifacts of the V-model are produced.  Those that
are created are circled with red ellipses, along with an annotation showing the
week number where the artifact is due for a full year (26 week) course.  The
week is when the Revision 0 draft of the document is due.  The Rev 0 documents
are graded, but the emphasis is on formative assessment, so their weight is low.
Documents are revised and re-graded at the end of the term (Rev1 Doc, Week 26).
This second evaluation is based on how well the students incorporated feedback
from the instructor, TAs and fellow students. The iteration allows students to
produce a higher quality document for their summative review (Rev 1). An
iterative process for a software capstone course is recommended by VanHanen and
Lehtinen~\cite{vanhanenSoftwareEngineeringProblems2014} to improve learning
outcomes. Although there are reasonably frequent interactions with the TA and
instructor, we do not follow the agile process recommended by
some~\cite{stettinaAcademicEducationSoftware2013,
bastarricaWhatCanStudents2017}.

In recognition of the value for teams of ``getting their hands dirty'',
a Proof of Concept (POC) Demo is scheduled for week 10. During this demo the
teams demonstrate the aspect of their project that is of most concern for
feasibility of the project, providing an opportunity to revise the project scope
if necessary. The Rev0 demo is expected to show off the final and complete
product. The teams rarely achieve this, but the push for Rev0, together with the
feedback they receive, allows them to improve their software for the final demo
(Rev 1 demo). The structure of the course is stable, having been offered in this
form for four years. The interventions described in the next sections are in the
context of this structure. 

\begin{figure}[h!]
  %\begin{center}
  \hspace{-0.6cm}
    {
      \includegraphics[width=1.1\columnwidth]{./figures/CourseStructure.drawio.pdf}
    }
    \caption{\label{Fig_VModel} V Model Used for Capstone Deliverables}
  %\end{center}
\end{figure}
% TODO - redraw the figure if time, and save as pdf
\subsection{Template Repository} \label{Sec_Template}

All teams start their project by using the same
%\href{https://github.com/smiths/capTemplate} 
\href{REDACTED Link} {GitHub template repository}. The template repo, summarized
in Figure~\ref{Fig_GitHubTemplate}, contains all the initial infrastructure each
team needs, including the folder structure, text-based template documents and
template issues. The goals of the template are to remove the time teams spend
building their project's infrastructure, and to standardize all the arbitrary
decisions, like folder and document names. The
standardization helps teams when doing peer reviews of each other's work and it
improves communication between teams, teaching assistants and instructor.
When students have a clear idea of the expectations, they should find it easier
to dive into their project.

\begin{figure}[h!]
  \begin{center}
    {
      \includegraphics[width=0.7\columnwidth]{./figures/GitHubTemplate}
    }
    \caption{\label{Fig_GitHubTemplate} GitHub Capstone Template}
  \end{center}
\end{figure}

The template documents are written in \LaTeX, although teams are allowed to redo
the template in another text-based format, like Markdown, if they wish. Besides
the advantage of separating document appearance from document content, the
text-based format facilitates tracking the productivity of the team members
through git commits, as discussed in Section~\ref{Sec_TeamContribMeasure}. The
documents correspond to the deliverables in Figure~\ref{Fig_VModel}. The
students can use any standard SRS template, including selecting one of the three
options given: SRS (a template for scientific computing
software~\cite{SmithAndLai2005}), SRS-Meyer (a template by Bertrand
Meyer~\cite{Meyer2022}) and SRS-Volere (the Volere
template~\cite{RobertsonAndRobertson1999Vol}).

For further standardization, the template repo includes
\href{Redact link}
%\href{https://github.com/smiths/capTemplate/tree/main/.github/ISSUE_TEMPLATE}
{four issue templates} for: 1) team meeting agendas; 2) TA-team meeting agendas;
3) supervisor-team meeting agendas; and, 4) lecture attendance.  In addition to
encouraging good organizational habits, the issues are also used to partly
measure the commitment of students to their teams, as discussed in the next
section.

\subsection{Team Contribution Measurement} \label{Sec_TeamContribMeasure}

To improve the distribution of the workload to all team members, we can take
advantage of the quantifiable productivity measures available from git and
GitHub.  The value of a metrics-based software engineering process is emphasized
by Conn~\cite{connReusableAcademicstrengthMetricsbased2004}, although they do
not list their recommended metrics.  In the current work the suggested metrics
for each team member are counting team meeting attendance and using GitHub
insights to count the commits to the main branch.  More complex metrics are
available, like lines of code, function points, use case points, object points,
and feature points~\cite{sudhakarMeasuringProductivitySoftware2012}, but by
default we keep things simple and standard for the teams.  However, if a team
desires more complex metrics, they can measure those alongside the required
ones. Teams are reminded that if they work together on something, they can use
co-author commits. Each team produces a summary table as part of their
\href{REDACT LINK}  
%\href{https://github.com/smiths/capTemplate/tree/main/docs/projMngmnt}
{performance reports}, which are produced before the three demonstrations: POC
demo, Rev0 demo and Rev1 demo (see Figure~\ref{Fig_VModel} for the timing of the
demos). In the performance report the team can record an explanation for why a
team member appears to perform poorly on any of the metrics. For instance, a
team member may have focused their commits on a branch that has not yet been
merged into main.

The teams set specific expectations for their team members in their team
charter. For instance, the team might have a rule that missing 20\% of the team
meetings before the proof of concept demonstration requires the offender to pick
up the coffee for the next team meeting. A more serious rule might be something
like, if a team member has less than 5\% of the total team commits before the
POC demo, the team will schedule a meeting with the course instructor to discuss
the problem. The encouraging feature of \emph{a priori} creation of rules is
that the difficult discussion happens while relationships among team members are
likely strong.  If a problem later occurs a team member doesn't have to muster
the courage to say that they are concerned with a colleague's performance,
instead they can point to the team charter and highlight the relevant, already
agreed upon rule.

The hope is that explicitly capturing productivity measures during the term will
reveal any problems with team collaboration.  Ideally the problems will be
revealed early and improved, but if the problem cannot be dealt with, at least
there will be enough data to assign a fair individual grade to all team members.
Although by default all team members share the same grade on a deliverable, this
can be multiplied by a ``team contribution factor'' if the data suggests this is
necessary for fairness.  The data is not just commits.  Any change in grades
needs to be supported by feedback from the TAs, feedback from supervisors,
instructor observations, and anonymous team surveys.  The team contribution
factor penalty is generally only applied after a team member has had an explicit
warning from the instructor.
=
\section{Preliminary Data} \label{SecPrelimData}

In this section we compare preliminary data for the 2022/23 and 2023/24 academic
years.  The two versions of the capstone course are similar.  Both follow the
V-model given in Section~\ref{Sec_Structure} and both use a Github template
(Section~\ref{Sec_Template}). The difference between them is that about a third
of the
way through the second course teams were asked to begin measuring team
performance metrics.  Neither course included a team charter.

\subsection{Timeline Comparison}

To measure the spread of work across time, we propose the following
metrics:

\begin{enumerate}
\item Daily commit graphs (examples for 2022/23 and 2023/24 are shown in
Figs.~\ref{Fig_22_23Timeline} \&~\ref{Fig_23_24Timeline}).
\item T-0 Proportion: The proportion of commits made on major deliverable due dates
\item T-2$\ldots$T-0 Proportion: The proportion of commits made on major deliverable due
      dates and in the two days prior to them.
\end{enumerate}

A summary of results in 2022-23 and 2023-24 is shown in Table~\ref{Tab:TimeMetrics}.
We observed similar results between the two years. There was a slight reduction in the 
T-0 and T-2...T-0 commits between these two
years, especially the T-0 commits. It is clear that there is still work to be done to
encourage spreading out work, but it is likely to be impossible to ever fully eliminate
this effect because of the nature of due dates in students' busy schedules, but
improvements in these metrics would show progress towards the goal of spreading out
work across time more effectively.

\begin{table}
\caption{Time-Spread Metrics Across Two Classes}\label{Tab:TimeMetrics}
\centering
\begin{tabular}{@{}lrrr@{}}
\toprule
\textbf{Metric}                      & \textbf{2022/23 Value} & \textbf{2023/24 Value} \\ \midrule
Total Commits                        & 6140                        & 5120                        \\
Total Days                           & 243                         & 244                         \\
T-0 Days                             & 10 (4.12\%)                 & 10 (4.10\%)                 \\
T-0 Commits                          & 1471 (23.96\%)              & 942 (18.40\%)               \\
T-2...T-0 Days                       & 30 (1.37\%)                 & 30 (1.37\%)                 \\
T-2...T-0 Commits                    & 2377 (38.71\%)              & 1872 (36.56\%)              \\ \bottomrule
\end{tabular}
\end{table}


\begin{figure}[h!]
\centering
\begin{tikzpicture}
\begin{axis}[
    ybar,
    bar width=0.25mm,
    width=0.5\textwidth,
    height=0.4\textwidth,
    symbolic x coords={2022-09-01, 2022-09-02, 2022-09-03, 2022-09-04, 2022-09-05, 2022-09-06, 2022-09-07, 2022-09-08, 2022-09-09, 2022-09-10, 2022-09-11, 2022-09-12, 2022-09-13, 2022-09-14, 2022-09-15, 2022-09-16, 2022-09-17, 2022-09-18, 2022-09-19, 2022-09-20, 2022-09-21, 2022-09-22, 2022-09-23, 2022-09-24, 2022-09-25, 2022-09-26, 2022-09-27, 2022-09-28, 2022-09-29, 2022-09-30, 2022-10-01, 2022-10-02, 2022-10-03, 2022-10-04, 2022-10-05, 2022-10-06, 2022-10-07, 2022-10-08, 2022-10-09, 2022-10-10, 2022-10-11, 2022-10-12, 2022-10-13, 2022-10-14, 2022-10-15, 2022-10-16, 2022-10-17, 2022-10-18, 2022-10-19, 2022-10-20, 2022-10-21, 2022-10-22, 2022-10-23, 2022-10-24, 2022-10-25, 2022-10-26, 2022-10-27, 2022-10-28, 2022-10-29, 2022-10-30, 2022-10-31, 2022-11-01, 2022-11-02, 2022-11-03, 2022-11-04, 2022-11-05, 2022-11-06, 2022-11-07, 2022-11-08, 2022-11-09, 2022-11-10, 2022-11-11, 2022-11-12, 2022-11-13, 2022-11-14, 2022-11-15, 2022-11-16, 2022-11-17, 2022-11-18, 2022-11-19, 2022-11-20, 2022-11-21, 2022-11-22, 2022-11-23, 2022-11-24, 2022-11-25, 2022-11-26, 2022-11-27, 2022-11-28, 2022-11-29, 2022-11-30, 2022-12-01, 2022-12-02, 2022-12-03, 2022-12-04, 2022-12-05, 2022-12-06, 2022-12-07, 2022-12-08, 2022-12-09, 2022-12-10, 2022-12-11, 2022-12-12, 2022-12-13, 2022-12-14, 2022-12-15, 2022-12-16, 2022-12-17, 2022-12-18, 2022-12-19, 2022-12-20, 2022-12-21, 2022-12-22, 2022-12-23, 2022-12-24, 2022-12-25, 2022-12-26, 2022-12-27, 2022-12-28, 2022-12-29, 2022-12-30, 2022-12-31, 2023-01-01, 2023-01-02, 2023-01-03, 2023-01-04, 2023-01-05, 2023-01-06, 2023-01-07, 2023-01-08, 2023-01-09, 2023-01-10, 2023-01-11, 2023-01-12, 2023-01-13, 2023-01-14, 2023-01-15, 2023-01-16, 2023-01-17, 2023-01-18, 2023-01-19, 2023-01-20, 2023-01-21, 2023-01-22, 2023-01-23, 2023-01-24, 2023-01-25, 2023-01-26, 2023-01-27, 2023-01-28, 2023-01-29, 2023-01-30, 2023-01-31, 2023-02-01, 2023-02-02, 2023-02-03, 2023-02-04, 2023-02-05, 2023-02-06, 2023-02-07, 2023-02-08, 2023-02-09, 2023-02-10, 2023-02-11, 2023-02-12, 2023-02-13, 2023-02-14, 2023-02-15, 2023-02-16, 2023-02-17, 2023-02-18, 2023-02-19, 2023-02-20, 2023-02-21, 2023-02-22, 2023-02-23, 2023-02-24, 2023-02-25, 2023-02-26, 2023-02-27, 2023-02-28, 2023-03-01, 2023-03-02, 2023-03-03, 2023-03-04, 2023-03-05, 2023-03-06, 2023-03-07, 2023-03-08, 2023-03-09, 2023-03-10, 2023-03-11, 2023-03-12, 2023-03-13, 2023-03-14, 2023-03-15, 2023-03-16, 2023-03-17, 2023-03-18, 2023-03-19, 2023-03-20, 2023-03-21, 2023-03-22, 2023-03-23, 2023-03-24, 2023-03-25, 2023-03-26, 2023-03-27, 2023-03-28, 2023-03-29, 2023-03-30, 2023-03-31, 2023-04-01, 2023-04-02, 2023-04-03, 2023-04-04, 2023-04-05, 2023-04-06, 2023-04-07, 2023-04-08, 2023-04-09, 2023-04-10, 2023-04-11, 2023-04-12, 2023-04-13, 2023-04-14, 2023-04-15, 2023-04-16, 2023-04-17, 2023-04-18, 2023-04-19, 2023-04-20, 2023-04-21, 2023-04-22, 2023-04-23, 2023-04-24, 2023-04-25, 2023-04-26, 2023-04-27, 2023-04-28, 2023-04-29, 2023-04-30, 2023-05-01},
    xmin=2022-09-01,
    xmax=2023-05-01,
    xtick=\empty,
    nodes near coords = {},
    nodes near coords align={vertical},
    ymin=0,
    ylabel={Commits},
    xlabel={Date},
    legend style={at={(0.5,-0.15)},anchor=north,legend columns=-1},
    ymajorgrids=false,
    grid style=dashed,
]

\addplot [draw=none,fill=blue,discard if not={Highlight}{None}
] table [
    x=Date,
    y=Commits,
    x index=0,col sep=comma
]{daily_commits_2022-23.csv};

\addplot [draw=none,fill=red,discard if not={Highlight}{Red},bar shift=-0.95mm
] table [
    x=Date,
    y=Commits,
    x index=0,col sep=comma
]{daily_commits_2022-23.csv};

\addplot [draw=none,fill=orange,discard if not={Highlight}{Orange},bar shift=-0.95mm
] table [
    x=Date,
    y=Commits,
    x index=0,col sep=comma
]{daily_commits_2022-23.csv};

\end{axis}
\end{tikzpicture}
\caption{Timeline of Commits for 2022--2023. Dates shown
in red are due dates for major written deliverables, and dates in orange are days where
presentations were scheduled.}\label{Fig_22_23Timeline}
\end{figure}
  

\begin{figure}[h!]
\centering
\begin{tikzpicture}
\begin{axis}[
    ybar,
    bar width=0.25mm,
    width=0.5\textwidth,
    height=0.4\textwidth,
    symbolic x coords={2023-09-01, 2023-09-02, 2023-09-03, 2023-09-04, 2023-09-05, 2023-09-06, 2023-09-07, 2023-09-08, 2023-09-09, 2023-09-10, 2023-09-11, 2023-09-12, 2023-09-13, 2023-09-14, 2023-09-15, 2023-09-16, 2023-09-17, 2023-09-18, 2023-09-19, 2023-09-20, 2023-09-21, 2023-09-22, 2023-09-23, 2023-09-24, 2023-09-25, 2023-09-26, 2023-09-27, 2023-09-28, 2023-09-29, 2023-09-30, 2023-10-01, 2023-10-02, 2023-10-03, 2023-10-04, 2023-10-05, 2023-10-06, 2023-10-07, 2023-10-08, 2023-10-09, 2023-10-10, 2023-10-11, 2023-10-12, 2023-10-13, 2023-10-14, 2023-10-15, 2023-10-16, 2023-10-17, 2023-10-18, 2023-10-19, 2023-10-20, 2023-10-21, 2023-10-22, 2023-10-23, 2023-10-24, 2023-10-25, 2023-10-26, 2023-10-27, 2023-10-28, 2023-10-29, 2023-10-30, 2023-10-31, 2023-11-01, 2023-11-02, 2023-11-03, 2023-11-04, 2023-11-05, 2023-11-06, 2023-11-07, 2023-11-08, 2023-11-09, 2023-11-10, 2023-11-11, 2023-11-12, 2023-11-13, 2023-11-14, 2023-11-15, 2023-11-16, 2023-11-17, 2023-11-18, 2023-11-19, 2023-11-20, 2023-11-21, 2023-11-22, 2023-11-23, 2023-11-24, 2023-11-25, 2023-11-26, 2023-11-27, 2023-11-28, 2023-11-29, 2023-11-30, 2023-12-01, 2023-12-02, 2023-12-03, 2023-12-04, 2023-12-05, 2023-12-06, 2023-12-07, 2023-12-08, 2023-12-09, 2023-12-10, 2023-12-11, 2023-12-12, 2023-12-13, 2023-12-14, 2023-12-15, 2023-12-16, 2023-12-17, 2023-12-18, 2023-12-19, 2023-12-20, 2023-12-21, 2023-12-22, 2023-12-23, 2023-12-24, 2023-12-25, 2023-12-26, 2023-12-27, 2023-12-28, 2023-12-29, 2023-12-30, 2023-12-31, 2024-01-01, 2024-01-02, 2024-01-03, 2024-01-04, 2024-01-05, 2024-01-06, 2024-01-07, 2024-01-08, 2024-01-09, 2024-01-10, 2024-01-11, 2024-01-12, 2024-01-13, 2024-01-14, 2024-01-15, 2024-01-16, 2024-01-17, 2024-01-18, 2024-01-19, 2024-01-20, 2024-01-21, 2024-01-22, 2024-01-23, 2024-01-24, 2024-01-25, 2024-01-26, 2024-01-27, 2024-01-28, 2024-01-29, 2024-01-30, 2024-01-31, 2024-02-01, 2024-02-02, 2024-02-03, 2024-02-04, 2024-02-05, 2024-02-06, 2024-02-07, 2024-02-08, 2024-02-09, 2024-02-10, 2024-02-11, 2024-02-12, 2024-02-13, 2024-02-14, 2024-02-15, 2024-02-16, 2024-02-17, 2024-02-18, 2024-02-19, 2024-02-20, 2024-02-21, 2024-02-22, 2024-02-23, 2024-02-24, 2024-02-25, 2024-02-26, 2024-02-27, 2024-02-28, 2024-02-29, 2024-03-01, 2024-03-02, 2024-03-03, 2024-03-04, 2024-03-05, 2024-03-06, 2024-03-07, 2024-03-08, 2024-03-09, 2024-03-10, 2024-03-11, 2024-03-12, 2024-03-13, 2024-03-14, 2024-03-15, 2024-03-16, 2024-03-17, 2024-03-18, 2024-03-19, 2024-03-20, 2024-03-21, 2024-03-22, 2024-03-23, 2024-03-24, 2024-03-25, 2024-03-26, 2024-03-27, 2024-03-28, 2024-03-29, 2024-03-30, 2024-03-31, 2024-04-01, 2024-04-02, 2024-04-03, 2024-04-04, 2024-04-05, 2024-04-06, 2024-04-07, 2024-04-08, 2024-04-09, 2024-04-10, 2024-04-11, 2024-04-12, 2024-04-13, 2024-04-14, 2024-04-15, 2024-04-16, 2024-04-17, 2024-04-18, 2024-04-19, 2024-04-20, 2024-04-21, 2024-04-22, 2024-04-23, 2024-04-24, 2024-04-25, 2024-04-26, 2024-04-27, 2024-04-28, 2024-04-29, 2024-04-30, 2024-05-01},
    xmin=2023-09-01,
    xmax=2024-05-01,
    xtick=\empty,
    nodes near coords = {},
    nodes near coords align={vertical},
    ymin=0,
    ylabel={Commits},
    xlabel={Date},
    legend style={at={(0.5,-0.15)},anchor=north,legend columns=-1},
    ymajorgrids=false,
    grid style=dashed,
]

\addplot [draw=none,fill=blue,discard if not={Highlight}{None}
] table [
    x=Date,
    y=Commits,
    x index=0,col sep=comma
]{daily_commits_2023-24.csv};

\addplot [draw=none,fill=red,discard if not={Highlight}{Red},bar shift=-0.95mm
] table [
    x=Date,
    y=Commits,
    x index=0,col sep=comma
]{daily_commits_2023-24.csv};

\addplot [draw=none,fill=orange,discard if not={Highlight}{Orange},bar shift=-0.95mm
] table [
    x=Date,
    y=Commits,
    x index=0,col sep=comma
]{daily_commits_2023-24.csv};

\end{axis}
\end{tikzpicture}
\caption{Timeline of Commits for 2023--2024. Dates shown
in red are due dates for major written deliverables, and dates in orange are days where
presentations were scheduled.}\label{Fig_23_24Timeline}
\end{figure}


\subsection{Measuring Fairness}

To measure the spread of work across team, the fairness of work distribution 
among teammates, we sought to compute a metric where values range from 0 to 1, 
and where values in between contain meaningful information.
Thus, we devised the following \textit{unfairness} metric where $C$:

$$
\text{unfairness}(C) = \frac{ \sum\limits_{c, x \in C, c > x} (c-x)}{(\left|C\right| -
1) \cdot \sum\limits_{c \in C} c}
$$

\noindent where $C$ is the multiset of teammates' numbers of commits to the 
repository.

The metric computes the sum of the difference between each teammate's commits
and those who committed less than them, normalized by the number of teammates
(excluding themselves) and the total number of commits. This yields a value
from 0 to 1, called the \textit{unfairness} metric, where:

\begin{itemize}
  \item 0 indicates that teammates did an equal amount of work
  \item 1 indicates that all the work was done by one teammate
  \item A value between 0 and 1 indicates the proportion of work 
        per person which could have been given to someone who did less work
\end{itemize}

Fairness is defined as $\text{fairness}(C) = 1 - \text{unfairness}(C)$.

For example, if a team with Persons A, B, and C did 10, 5 and 5 commits respectively, then
$\text{unfairness}(\{10,5,5\}) = 0.25$. This is because on average Person A did 5 more commits than
their teammates and thus these 5 commits (out of 20) are considered \textit{unfair work}. The fairness 
value is thus 0.75.

Figs.~\ref{Fig:Fairness22-23} and~\ref{Fig:Fairness23-24} show the fairness values for 
teams in 2022/23 and 2023/24 respectively.

In the future, we will experiment with applying this metric to things other than commits (e.g.~lines of
code written, issues created/closed, etc.), as well as investigating a correlation between
lower fairness values and perceived unfairness according to the teammates themselves. Another
research question would be if having live access to this fairness metric encourages teams to share
work more evenly or simply encourages them to ``game the system'' to increase the value.

\begin{figure}[h]
\centering
\begin{tikzpicture}
\begin{axis}[
    ybar,
    bar width=0.3cm,
    width=0.5\textwidth,
    height=0.4\textwidth,
    symbolic x coords={Flick-Picker/full-stack, marlon4dashen/Hairesthetics, arkinmodi/project-sayyara, jeff-rey-wang/utrition, mehtaj8/Greenway, Tamas-Leung/CodeChamp, HKanwal/kapstone, paezha/PyERT-BLACK, BillNguyen1999/REVITALIZE, RutheniumVI/UnderTree, agentvv/MTOBridge, NicLobo/Capstone-yoGERT, brandonduong/Farming-Matters},
    xtick=\empty,
    nodes near coords = {},
    nodes near coords align={vertical},
    ymin=0,
    ymax=1,
    ylabel={Fairness},
    xlabel={Team},
    enlarge x limits=0.1,
    legend style={at={(0.5,-0.15)},anchor=north,legend columns=-1},
    ymajorgrids=false,
    grid style=dashed,
]
\addplot coordinates {
    (Flick-Picker/full-stack, 0.3562048588312541)
    (marlon4dashen/Hairesthetics, 0.4509090909090909)
    (arkinmodi/project-sayyara, 0.5239228125826064)
    (jeff-rey-wang/utrition, 0.541871921182266)
    (mehtaj8/Greenway, 0.5447537473233405)
    (Tamas-Leung/CodeChamp, 0.5865051903114187)
    (HKanwal/kapstone, 0.6448184233835252)
    (paezha/PyERT-BLACK, 0.6550335570469799)
    (BillNguyen1999/REVITALIZE, 0.6649842271293376)
    (RutheniumVI/UnderTree, 0.7277227722772277)
    (agentvv/MTOBridge, 0.751219512195122)
    (NicLobo/Capstone-yoGERT, 0.8527835051546392)
    (brandonduong/Farming-Matters, 0.8534278959810875)
};
\end{axis}
\end{tikzpicture}
\caption{Fairness of Commits Per Team 2022/23 [n=13] (Mean: 0.63, Stddev: 0.15)}\label{Fig:Fairness22-23}
\end{figure}

\begin{figure}[h]
\centering
\begin{tikzpicture}
\begin{axis}[
    ybar,
    bar width=0.3cm,
    width=0.5\textwidth,
    height=0.4\textwidth,
    symbolic x coords={r-yeh/grocery-spending-tracker, DangJustin/CapstoneProject, Tusharagg1/chest-x-ray-ai, d-akselrod/SweatSmart, InfiniView-AI/MotionMingle, stanreee/sign-language-learning, RishiVaya/capstone-12, MichaelBreau/nlp-mentalhealth, SammyG7/Mac-AR, beatlepie/4G06CapstoneProjectTeam2, katrina799/4G06CapstoneProjectT5},
    xtick=\empty,
    nodes near coords = {},
    nodes near coords align={vertical},
    ymin=0,
    ymax=1,
    ylabel={Fairness},
    xlabel={Team},
    enlarge x limits=0.1,
    legend style={at={(0.5,-0.15)},anchor=north,legend columns=-1},
    ymajorgrids=false,
    grid style=dashed,
]
\addplot coordinates {
    (r-yeh/grocery-spending-tracker, 0.3200145958766648)
    (DangJustin/CapstoneProject, 0.40943193997856375)
    (Tusharagg1/chest-x-ray-ai, 0.42666666666666664)
    (d-akselrod/SweatSmart, 0.5186666666666666)
    (InfiniView-AI/MotionMingle, 0.525830258302583)
    (stanreee/sign-language-learning, 0.5411013567438148)
    (RishiVaya/capstone-12, 0.6203860480866915)
    (MichaelBreau/nlp-mentalhealth, 0.6298472385428907)
    (SammyG7/Mac-AR, 0.6379532486655624)
    (beatlepie/4G06CapstoneProjectTeam2, 0.6618181818181819)
    (katrina799/4G06CapstoneProjectT5, 0.7661417322834646)
};
\end{axis}
\end{tikzpicture}
\caption{Fairness of Commits Per Team 2023/24 [n=11] (Mean: 0.55, Stddev: 0.13)}\label{Fig:Fairness23-24}
\end{figure}

\section{Proposed Experiment} \label{SecProposedExperiment}

blurb

\subsection{Experiment}

Start with research questions.

entry and exit surveys

Collect the same data as in Section~\ref{SecPrelimData} and conduct focus groups
in all three CAS capstone courses (SE, CS and TRON).  

\subsection{Threats to validity}

\begin{itemize}
    \item Using commits to measure productivity is possibly too simple a
    measure.  Some team members might commit small deltas of work, while others
    only commit when they complete a major change.  Going forward, productivity
    may be measured by a count of work
    events~\cite{saadatAnalyzingProductivityGitHub2020}, where a work event is a
    push, merged pull request, issue comment or pull request.  review comment
    \item Multiple changes are made to the course, so it is difficult to
    determine which change influences the student behaviour.  The focus group
    should hopefully tease that out.
    \item Comparing different courses with different instructors, different
    backgrounds for students, etc.
    \item Not a controlled experiment - introducing more than one change into
    the course.  The changes are related because the productivity metrics would
    not be possible without a version control system.
    \item The interventions proposed here might behave differently for a
    capstone course that follows a different structure
    (Section~\ref{Sec_Structure}).

\end{itemize}

\section{Concluding Remarks} \label{SecConclusions}

The template presented here is for the capstone course under discussion; the
template could be forked and modified to match the needs of a different capstone
course.

\section{Data Availability}
The raw data used to make these graphs is available on the paper's 
\href{REDACTED Link}{GitHub repository}, along with the scripts used to generate
the data.

\bibliographystyle{IEEEtran}
\bibliography{SmithEtAl2024_CSEEnT}

\end{document}
