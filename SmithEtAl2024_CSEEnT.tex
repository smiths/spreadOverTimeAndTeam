\documentclass[10pt, conference]{IEEEtran}

\usepackage[pdftex]{graphicx}
\usepackage[cmex10]{amsmath}
\usepackage{amsfonts}

\usepackage{color}
% \usepackage{hyperref}
% \hypersetup{colorlinks=true,
%     linkcolor=blue,
%     citecolor=blue,
%     filecolor=blue,
%     urlcolor=blue,
%     unicode=false}
% \urlstyle{same}

\usepackage{tabularx}
\usepackage{booktabs}
\usepackage{siunitx}
\usepackage{subfig}
\usepackage{paralist}
\usepackage{colortbl}
\usepackage{listings}

\usepackage{enumitem}

\newif\ifcomments\commentstrue

\ifcomments
\newcommand{\authornotation}[3]{\textcolor{#1}{[#3 ---#2]}}
\newcommand{\todo}[1]{\textcolor{red}{[TODO: #1]}}
\else
\newcommand{\authornotation}[3]{}
\newcommand{\todo}[1]{}
\fi

\newcommand{\wss}[1]{\authornotation{blue}{SS}{#1}}
\newcommand{\ms}[1]{\authornote{cyan}{MS}{#1}}

\newcommand{\progname}{SFS}
\newcommand{\colAwidth}{0.13\textwidth}
\newcommand{\colBwidth}{0.84\textwidth}

\begin{document}

\title{A Software Engineering Capstone Infrastructure that Encourages Spreading
Work Over Time and Team}

\author{\IEEEauthorblockN{Spencer Smith, Christopher Schankula, Lucas Dutton and Christopher Anand}
\IEEEauthorblockA{Computing and Software Department\\
McMaster University, Canada\\
Email: smiths@mcmaster.ca, schankuc@mcmaster.ca, duttonl@mcmaster.ca, anandc@mcmaster.ca}
% \and
% \IEEEauthorblockN{Sumanth Shankar}
% \IEEEauthorblockA{Mechanical Engineering Department\\
% McMaster University, Canada\\
% Email: shankar@mcmaster.ca }
}

\maketitle
  
\begin{abstract}

How can instructors facilitate spreading out the work in a software engineering
or computer science capstone course across time and between team members?
Currently teams often compromise the quality of their learning experience by
frantically working before each deliverable.  Some team members further
compromise learning by not contributing their fair share to the team effort. To
mitigate these problems, we propose using a GitHub template that contains all
the infrastructure that is reusable for all teams.  All teams will populate
their initial repositories with the same folder structure, text-based template
documents and template issues. In addition, we propose that at the start of the
term each team identify specific quantifiable productivity metrics that they
will monitor for each team member, such as the count of meetings attended,
issues closed and number of commits.  Initial data suggests that these steps
have an impact.  In 2022/23 we observed 50\% of commits happening within 3 days
of due dates.  After partial introducing the above ideas in 2023/24, this number
improved to 37\%. Going forward we propose a proper experiment where commit data
and interview data is compared between teams that use the proposed interventions
and those that do not.

\end{abstract}

\begin{IEEEkeywords}
software engineering; capstone; template repository
\end{IEEEkeywords}

\section{Introduction} \label{SecIntro}

The workload for a software engineering or computer science capstone team
project is often unevenly distributed over time and between team members.  Teams
typically work in frantic bursts of activity right before a deadline and then
cease almost all activity until their next deadline.  These work habits
compromise the learning objectives of the course because the students do not
have time to properly plan their activities or reflect on their work.  The
uneven distribution of effort between team mates is also problematic.  Some
students take on an unfair share of the work, causing them stress and possibly
hurting their experience in other courses, while others miss important learning
opportunities.  How can instructors mitigate these problems?

To address them, need to first think about why the problems exist.  Not the same
as the workplace.  Other pressures on students.  Not sure of expectations.  Not
sure where to begin.  Peer pressure and social interactions that make it
challenging to take charge of the group, or criticize other group members.
[There must be some literature that talks about the challenges for student
teamwork, teamwork in SE, teamwork for capstone projects, teamwork for SE
capstone projects]

Ideas on what we can do about it at an abstract level - the forces we can use to
direct students.  We have grades and we have structure of the course and
expectations.

Overview of ideas.

Roadmap of paper.

Making sure that BibTex is working \cite{Smith2005}.

\section{Literature Review} \label{SecLitReview}
May not need this if the literature is covered in the introduction.

\section{Proposed Infrastructure} \label{SecInfrastruct}

roadmap blurb

\subsection{Structure and Timeline}

figure showing the V model and the expected deliverables.  The ideas in this
paper could work for other structures, but this is the one adopted.

\subsection{Template Repository}

\subsection{Team Measures of Productivity}

\section{Preliminary Data} \label{SecPrelimData}

Look at commits over time, and possibly lines (removing outliers) of code over
time

\section{Proposed Experiment} \label{SecProposedExperiment}

Start with research questions.

Collect the same data as in Section~\ref{SecPrelimData} and conduct focus groups
in all three CAS capstone courses (SE, CS and TRON).  

Threats to validity:

\begin{itemize}
    \item Multiple changes are made to the course, so it is difficult to
    determine which change influences the student behaviour.  The focus group
    should hopefully tease that out.
    \item Comparing different courses with different instructors, different
    backgrounds for students, etc.
    \item etc.
\end{itemize}

\section{Concluding Remarks} \label{SecConclusions}

\section*{Acknowledgements}

If any.

\bibliographystyle{IEEEtran}
\bibliography{SmithEtAl2024_CSEEnT}

\end{document}