\documentclass[10pt, conference]{IEEEtran}

\usepackage[pdftex]{graphicx}
\usepackage[cmex10]{amsmath}
\usepackage{amsfonts}

\usepackage{color}
% \usepackage{hyperref}
% \hypersetup{colorlinks=true,
%     linkcolor=blue,
%     citecolor=blue,
%     filecolor=blue,
%     urlcolor=blue,
%     unicode=false}
% \urlstyle{same}

\usepackage{tabularx}
\usepackage{booktabs}
\usepackage{siunitx}
\usepackage{subfig}
\usepackage{paralist}
\usepackage{colortbl}
\usepackage{listings}

\usepackage{enumitem}

\newif\ifcomments\commentstrue

\ifcomments
\newcommand{\authornotation}[3]{\textcolor{#1}{[#3 ---#2]}}
\newcommand{\todo}[1]{\textcolor{red}{[TODO: #1]}}
\else
\newcommand{\authornotation}[3]{}
\newcommand{\todo}[1]{}
\fi

\newcommand{\wss}[1]{\authornotation{blue}{SS}{#1}}
\newcommand{\ms}[1]{\authornote{cyan}{MS}{#1}}

\newcommand{\progname}{SFS}
\newcommand{\colAwidth}{0.13\textwidth}
\newcommand{\colBwidth}{0.84\textwidth}

\begin{document}

\title{A Software Engineering Capstone Infrastructure that Encourages Spreading
Work Over Time and Team}

\author{\IEEEauthorblockN{Spencer Smith, Christopher Schankula, Lucas Dutton and Christopher Anand}
\IEEEauthorblockA{Computing and Software Department\\
McMaster University, Canada\\
Email: smiths@mcmaster.ca, schankuc@mcmaster.ca, duttonl@mcmaster.ca, anandc@mcmaster.ca}
% \and
% \IEEEauthorblockN{Sumanth Shankar}
% \IEEEauthorblockA{Mechanical Engineering Department\\
% McMaster University, Canada\\
% Email: shankar@mcmaster.ca }
}

\maketitle
  
\begin{abstract}

Problem. Proposed Solution. Initial Results. Proposed Experiment.

\end{abstract}

\begin{IEEEkeywords}
software engineering; capstone; template repository
\end{IEEEkeywords}

\section{Introduction} \label{SecIntro}

Making sure that BibTex is working \cite{Smith2005}.

\section{Literature Review} \label{SecLitReview}

\section{Proposed Infrastructure} \label{SecInfrastruct}

\section{Preliminary Data} \label{SecPrelimData}

Look at commits over time, and possibly lines (removing outliers) of code over
time

\section{Proposed Experiment} \label{SecProposedExperiment}

Collect the same data as in Section~\ref{SecPrelimData} and conduct focus groups
in all three CAS capstone courses (SE, CS and TRON).

\section{Concluding Remarks} \label{SecConclusions}

\section*{Acknowledgements}

\bibliographystyle{IEEEtran}
\bibliography{SmithEtAl2024_CSEEnT}

\end{document}