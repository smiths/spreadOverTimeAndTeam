%\documentclass[poster, a1, landscape, plainboxedsections]{sciposter}
\documentclass[poster, a0, plainboxedsections]{sciposter}

\usepackage{multicol}
\usepackage{amsmath}
\usepackage{amsfonts}
\usepackage{graphicx}
\usepackage{booktabs}
\usepackage{tikz}
\usepackage{pgfplots}
\usepackage{pgfplotstable} % For importing data from a csv


%\leftlogo[1.1]{McMaster_BlackLogo.pdf}
\leftlogo[1.0]{eng_logo.png}
\rightlogo[1.0]{STaBLLogoWS.png}
\title{A Software Engineering Capstone Infrastructure that Encourages Spreading
Work Over Time and Team}
\author{Spencer Smith, Christopher Schankula, Lucas Dutton and Christopher Anand}
\institute{Computing and Software Department, McMaster University}
\email{smiths@mcmaster.ca}

\begin{document}

\conference{{\bf CSEE\&T 2025}, IEEE Conference on Software Engineering
Education and Training, Apr 28--29, Ottawa, ON, Canada; Poster 96 (S)M}

\maketitle

\setlength{\columnseprule}{0pt}
\begin{multicols}{2}

\PARstart {H}{ow} can instructors spread out the work in computing capstone
courses across time and among team members? 
\begin{itemize}
\item Use a GitHub template that contains the initial infrastructure, including
the folder structure, text-based template documents and template issues
\item Have each team begin the year by identifying specific quantifiable
individual productivity metrics for monitoring, such as the count of meetings
attended, issues closed and number of commits
\end{itemize}

\noindent To measure the effectiveness of our intervention on the distribution of work, we
introduce a new fairness metric.

\section{Proposed Process and Infrastructure} \label{SecPropInfrastruc}

\begin{figure}[!h]
\includegraphics[width=1.0\linewidth]{../figures/CourseStructure.drawio.pdf}
\caption{\label{Fig_VModel} V Model Used for Capstone Deliverables}
\label{FigStructure}
\end{figure}

\begin{itemize}
\item 8-month capstone course, teams of 4--5 following a V-model process
\item Teams propose a team charter that includes specific quantifiable expectations and
consequences
\item Team member productivity is reported to the instructor
\item A template Github repository is used for standardization and to save time
\end{itemize}

\begin{tikzpicture}[remember picture,overlay]
\node [xshift=-4.5cm,yshift=-3.5cm] at (current page.center)
{
\includegraphics[width=0.1\linewidth]{../figures/captemplate.png}
};
\end{tikzpicture}

\section{Time Spread Before and After Templates}

\begin{figure}[h!]
\centering
\includegraphics[width=0.9\linewidth]{../figures/HistCommits2022-23.pdf}
\caption{Commits for 2022--2023.}\label{Fig_22_23Timeline}
\end{figure}

\begin{figure}[h!]
\centering
\includegraphics[width=0.9\linewidth]{../figures/HistCommits2023-24.pdf}
\caption{Commits for 2023--2024.}\label{Fig_23_24Timeline}
\end{figure}

\begin{itemize}
\item Data shows the benefit of partial implementation of the proposed
intervention
\item Still room for improvement
\item Full elimination of the deadline effect is unlikely because of the nature
of due dates in students' busy schedules
\end{itemize}

\begin{table}
\caption{T0 on deadline, T2 up to two days prior}
\centering
\begin{tabular}{@{}lrrr@{}}
\toprule
\textbf{Metric} & \textbf{2022/23 Value} & \textbf{2023/24 Value} \\ 
\midrule
Total Commits & 6140 & 5120 \\
T0 Commits & 1471 (23.96\%) & 942 (18.40\%) \\
T2 Commits & 2377 (38.71\%) & 1872 (36.56\%) \\ 
\bottomrule
\end{tabular}
\end{table}

\section*{Team Fairness Before and After Templates}

$$
\text{fairness}(C) = 1 - \frac{ \sum\limits_{c, x \in C, c > x} (c-x)}{(\left|C\right| -
1) \cdot \sum\limits_{c \in C} c}
$$

\noindent where $C$ is the multiset of teammates' numbers of commits to the 
repository.

\begin{itemize}
\item The index computes one minus the sum of the difference between each teammate's
commits and those who committed less than them, normalized by the number of
teammates (excluding themselves) and the total number of commits
\item The index value is between 1 and 0
\item Call it the \textit{fairness} index
\end{itemize}

\begin{figure}[h]
\centering
\includegraphics[width=1.0\linewidth]{../figures/FairnessCommits_22_23.pdf}
\caption{Fairness of Commits 2022/23 [n=13; Team Fairness Mean: 0.63, Stddev: 0.15; Time Fairness Mean: 0.12, Stddev: 0.05; Correlation: -0.16]}\label{Fig:Fairness2022/23}
\end{figure}

\begin{figure}[h]
\centering
\includegraphics[width=1.0\linewidth]{../figures/FairnessCommits_23_24.pdf}
\caption{Fairness of Commits 2023/24 [n=11; Team Fairness Mean: 0.55, Stddev: 0.13; Time Fairness Mean: 0.14, Stddev: 0.04; Correlation: 0.15]}\label{Fig:Fairness2023/24}
\end{figure}

In the future
\begin{itemize}
    \item Experiment with applying new index to other metrics like lines of
    code, issues created/closed, pull requests created/merged, etc.
    \item Use multiple metrics at once, similar to the multi-Jain fairness
    index 
    \item Investigate correlation between lower fairness values and perceived
    unfairness by the teammates
    \item Investigate whether live access fairness index encourages teams to share
    work or simply encourages them to ``game the system''
\end{itemize}

\section{Conclusions}

\PARstart {P}{reliminary} data suggests that templates, team charters and
productivity monitoring may have an impact.  

\begin{itemize}
\item In the base year teams put off work until the last minute
\item Before intervention we observed 24\% of commits happening on the due dates
\item After partially introducing the proposed interventions this number
improved to 18\%
\end{itemize}

Going forward, we propose
an experiment where commit data and interview data is compared between teams
that use the proposed interventions and those that do not.

\begin{figure}[htbp]
  \centering
  \includegraphics[height=3.2cm,trim={0 0 28.5cm 0},clip]{nserc-logo.jpg}
  \hspace{0.5cm}
  \includegraphics[height=3.2cm,trim={16.5cm 0 0 0},clip]{ontario@2x-print.png}
  \hspace{0.5cm}
  \includegraphics[height=3.2cm,trim={0 0.5cm 0 0.48cm},clip]{outreachlogo.png}  
\end{figure}
~\\

\begin{tikzpicture}[remember picture,overlay]
\node [xshift=30cm,yshift=-55.5cm] at (current page.center)
{
\includegraphics[width=0.15\linewidth]{bit.ly_3EyObhv.png}
};
\end{tikzpicture}

\end{multicols}

\end{document}